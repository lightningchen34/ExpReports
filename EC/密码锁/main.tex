%!TEX program=xelatex
\documentclass[a4paper]{article}
\usepackage{ctex}
\usepackage{geometry}
\usepackage{graphicx}
\usepackage{float}
\usepackage{listings}
\usepackage{xcolor}
\lstset{
    numbers=left, 
    numberstyle= \tiny, 
    keywordstyle= \color{ blue!70},
    commentstyle= \color{red!50!green!50!blue!50}, 
    frame=shadowbox,
    rulesepcolor= \color{ red!20!green!20!blue!20} ,
    escapeinside=``,
    framexleftmargin=2em
}

\geometry{left=3.0cm,right=3.0cm,top=3.5cm,bottom=3.5cm}

\title{串行密码锁实验报告}
\author{2017011341, 陈旭}
\date{2019 年 5 月}

\begin{document}

\maketitle

\section{实验目的}

    \begin{itemize}
        \item 学习使用状态机控制电路工作,在不同状态下完成相应的功能;
        \item 进一步掌握时序逻辑电路的基本分析和设计方法;
        \item 学会利用仿真软件实现对数字电路的逻辑功能进行验证和分析。
    \end{itemize}

\section{实验内容}

    \begin{itemize}
        \item 设计一个 4 位 16 进制串行密码锁,支持:设置密码、验证密码解锁。
        \item 提高部分:管理员万用密码、错误次数警报。
    \end{itemize}

\section{代码与分析}

    \subsubsection{代码}

    \begin{lstlisting}
library ieee;
use ieee.std_logic_1164.all;
use ieee.std_logic_arith.all;
use ieee.std_logic_unsigned.all;

entity lock is 
  port(
    clock : in std_logic;
    code : in std_logic_vector(3 downto 0);
    mode : in std_logic_vector(1 downto 0);
    clk, rst : in std_logic;
    unlock, setting : out std_logic;
    alarm, error : buffer std_logic;
    lights : out std_logic_vector(3 downto 0)
  );
  type T is array (3 downto 0) of integer;
end lock;

architecture lock of lock is
  signal password : T;
  signal manager : T := (9, 6, 8, 4);
  signal state : integer := 0;
  signal error_count : integer := 0;
  signal click : std_logic;
begin
  process(clock)
  begin
    if clock'event and clock='1' then
      click <= clk;
    end if;
  end process;
  
  process(click, rst)
  begin
    if click'event and click='1' then
      if mode="00" then
        case state is
          when 0 => password(0) <= CONV_INTEGER(code); state <= 1;
          when 1 => password(1) <= CONV_INTEGER(code); state <= 2;
          when 2 => password(2) <= CONV_INTEGER(code); state <= 3;
          when 3 => password(3) <= CONV_INTEGER(code); state <= 4;
          when 4 => 
            unlock <= '0';
            if CONV_INTEGER(code) = manager(0) then
              state <= 5;
            else
              state <= 4;
            end if;
          when 5 => 
            if CONV_INTEGER(code) = manager(1) then
              state <= 6;
            else
              state <= 4;
            end if;
          when 6 => 
            if CONV_INTEGER(code) = manager(2) then
              state <= 7;
            else
              state <= 4;
            end if;
          when 7 => 
            if CONV_INTEGER(code) = manager(3) then
              state <= 18;
              alarm <= '0';
              error_count <= 0;
            else
              state <= 4;
            end if;
          when 18 => 
            state <= 0;
            when others => state <= 4; unlock <= '0'; error <= '0';
        end case;
      elsif mode="01" then
        case state is
          when 0 => password(0) <= CONV_INTEGER(code); state <= 1;
          when 1 => password(1) <= CONV_INTEGER(code); state <= 2;
          when 2 => password(2) <= CONV_INTEGER(code); state <= 3;
          when 3 => password(3) <= CONV_INTEGER(code); state <= 8;
          when 8 => 
            unlock <= '0';
            if alarm='0' then
              if CONV_INTEGER(code) = password(0) and 
                  CONV_INTEGER(code) = manager(0) then
                state <= 9;
              elsif CONV_INTEGER(code) = password(0) then
                state <= 12;
              elsif CONV_INTEGER(code) = manager(0) then
                state <= 15;
              else
                state <= 8;
                error <= '1';
                if error_count + 1 = 3 then
                  alarm <= '1';
                  error_count <= 0;
                else
                  error_count <= error_count + 1;
                end if;
              end if;
            end if;
          when 9 => 
            if CONV_INTEGER(code) = password(1) and 
                CONV_INTEGER(code) = manager(1) then
              state <= 10;
            elsif CONV_INTEGER(code) = password(1) then
              state <= 13;
            elsif CONV_INTEGER(code) = manager(1) then
              state <= 16;
            else
              state <= 8;
              error <= '1';
              if error_count + 1 = 3 then
                alarm <= '1';
                error_count <= 0;
              else
                error_count <= error_count + 1;
              end if;
            end if;
          when 10 => 
            if CONV_INTEGER(code) = password(2) and 
                CONV_INTEGER(code) = manager(2) then
              state <= 11;
            elsif CONV_INTEGER(code) = password(2) then
              state <= 14;
            elsif CONV_INTEGER(code) = manager(2) then
              state <= 17;
            else
              state <= 8;
              error <= '1';
              if error_count + 1 = 3 then
                alarm <= '1';
                error_count <= 0;
              else
                error_count <= error_count + 1;
              end if;
            end if;
          when 11 => 
            if CONV_INTEGER(code) = password(3) or 
                CONV_INTEGER(code) = manager(3) then
              state <= 8;
              unlock <= '1';
              error_count <= 0;
              error <= '0';
            else
              state <= 8;
              error <= '1';
              if error_count + 1 = 3 then
                alarm <= '1';
                error_count <= 0;
              else
                error_count <= error_count + 1;
              end if;
            end if;
          when 12 =>
            if CONV_INTEGER(code) = password(1) then
              state <= 13;
            else
              state <= 8;
              error <= '1';
              if error_count + 1 = 3 then
                alarm <= '1';
                error_count <= 0;
              else
                error_count <= error_count + 1;
              end if;
            end if;
          when 13 =>
            if CONV_INTEGER(code) = password(2) then
              state <= 14;
            else
              state <= 8;
              error <= '1';
              if error_count + 1 = 3 then
                alarm <= '1';
                error_count <= 0;
              else
                error_count <= error_count + 1;
              end if;
            end if;
          when 14 =>
            if CONV_INTEGER(code) = password(3) then
              state <= 8;
              unlock <= '1';
              error_count <= 0;
              error <= '0';
            else
              state <= 8;
              error <= '1';
              if error_count + 1 = 3 then
                alarm <= '1';
                error_count <= 0;
              else
                error_count <= error_count + 1;
              end if;
            end if;
          when 15 =>
            if CONV_INTEGER(code) = manager(1) then
              state <= 16;
            else
              state <= 8;
              error <= '1';
              if error_count + 1 = 3 then
                alarm <= '1';
                error_count <= 0;
              else
                error_count <= error_count + 1;
              end if;
            end if;
          when 16 =>
            if CONV_INTEGER(code) = manager(2) then
              state <= 17;
            else
              state <= 8;
              error <= '1';
              if error_count + 1 = 3 then
                alarm <= '1';
                error_count <= 0;
              else
                error_count <= error_count + 1;
              end if;
            end if;
          when 17 =>
            if CONV_INTEGER(code) = manager(3) then
              state <= 8;
              unlock <= '1';
              error_count <= 0;
              error <= '0';
            else
              state <= 8;
              error <= '1';
              if error_count + 1 = 3 then
                alarm <= '1';
                error_count <= 0;
              else
                error_count <= error_count + 1;
              end if;
            end if;
          when others => state <= 8; unlock <= '0'; error <= '0';
        end case;
      end if;
    end if;
  end process;
  
  process(state)
  begin
    if state < 4 then
      setting <= '1';
    else
      setting <= '0';
    end if;
    case state is
      when 0 => 
        lights(0) <= '1'; lights(1) <= '0'; 
        lights(2) <= '0'; lights(3) <= '0';
      when 1 => 
        lights(0) <= '1'; lights(1) <= '1'; 
        lights(2) <= '0'; lights(3) <= '0';
      when 2 => 
        lights(0) <= '1'; lights(1) <= '1'; 
        lights(2) <= '1'; lights(3) <= '0';
      when 3 => 
        lights(0) <= '1'; lights(1) <= '1'; 
        lights(2) <= '1'; lights(3) <= '1';
      when 4 => 
        lights(0) <= '1'; lights(1) <= '0'; 
        lights(2) <= '0'; lights(3) <= '0';
      when 5 => 
        lights(0) <= '1'; lights(1) <= '1'; 
        lights(2) <= '0'; lights(3) <= '0';
      when 6 => 
        lights(0) <= '1'; lights(1) <= '1'; 
        lights(2) <= '1'; lights(3) <= '0';
      when 7 => 
        lights(0) <= '1'; lights(1) <= '1'; 
        lights(2) <= '1'; lights(3) <= '1';
      when 8 => 
        lights(0) <= '1'; lights(1) <= '0'; 
        lights(2) <= '0'; lights(3) <= '0';
      when 9 => 
        lights(0) <= '1'; lights(1) <= '1'; 
        lights(2) <= '0'; lights(3) <= '0';
      when 10 => 
        lights(0) <= '1'; lights(1) <= '1'; 
        lights(2) <= '1'; lights(3) <= '0';
      when 11 => 
        lights(0) <= '1'; lights(1) <= '1'; 
        lights(2) <= '1'; lights(3) <= '1';
      when 12 => 
        lights(0) <= '1'; lights(1) <= '1'; 
        lights(2) <= '0'; lights(3) <= '0';
      when 13 => 
        lights(0) <= '1'; lights(1) <= '1'; 
        lights(2) <= '1'; lights(3) <= '0';
      when 14 => 
        lights(0) <= '1'; lights(1) <= '1'; 
        lights(2) <= '1'; lights(3) <= '1';
      when 15 => 
        lights(0) <= '1'; lights(1) <= '1'; 
        lights(2) <= '0'; lights(3) <= '0';
      when 16 => 
        lights(0) <= '1'; lights(1) <= '1'; 
        lights(2) <= '1'; lights(3) <= '0';
      when 17 => 
        lights(0) <= '1'; lights(1) <= '1'; 
        lights(2) <= '1'; lights(3) <= '1';
      when 18 => 
        lights(0) <= '0'; lights(1) <= '0'; 
        lights(2) <= '0'; lights(3) <= '0';
      when others => NULL;
    end case;
  end process;
end lock;
    \end{lstlisting}

    \subsubsection{分析}
        \par 状态介绍:
        \begin{itemize}
            \item 0: 开始设置密码
            \item 1: 设置第二位密码
            \item 2: 设置第三位密码
            \item 3: 设置第四位密码
            \item 4: 管理模式,开始输入管理员密码
            \item 5: 输入第二位管理员密码
            \item 6: 输入第三位管理员密码
            \item 7: 输入第四位管理员密码
            \item 8: 解锁模式,开始输入解锁密码
            \item 9: 输入第二位解锁密码,可接受管理员密码与用户密码
            \item 10: 输入第三位解锁密码,可接受管理员密码与用户密码
            \item 11: 输入第四位解锁密码,可接受管理员密码与用户密码
            \item 12: 输入第二位解锁密码,可接受用户密码
            \item 13: 输入第三位解锁密码,可接受用户密码
            \item 14: 输入第四位解锁密码,可接受用户密码
            \item 15: 输入第二位解锁密码,可接受管理员密码
            \item 16: 输入第三位解锁密码,可接受管理员密码
            \item 17: 输入第四位解锁密码,可接受管理员密码
            \item 18: 按下按键开始设置密码,或切换到验证密码状态
        \end{itemize}

        \par 启动时开始设置密码,设置密码后根据 Mode 进入管理员模式(状态 4)或验证模式(状态 8);管理员模式下,成功验证管理员模式将进入状态 18,确认后可修改密码,也可以不进行确认直接转到验证模式;验证模式下,可输入管理员密码与用户密码进行解锁,若解锁失败累计失败次数,且回到状态 8,失败次数过多则无法再解锁,只能切换到管理员模式用管理员密码清除失败次数。

\section{实验小结}

    \par 本次实验给我最大的收获就是用 VHDL 来实现状态机,通过这个实验,让我更加熟悉状态机的理论,用 VHDL 实现状态机来解决具体问题。
    \par 一开始没有考虑到用管理员密码解锁的情况,到了检查的时候才进行修改,浪费了一些时间。因此在定义问题时就需要把问题考虑清楚,避免再次遇到这这样的情况。
    

\end{document}