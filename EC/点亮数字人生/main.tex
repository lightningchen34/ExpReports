%!TEX program=xelatex
\documentclass[a4paper]{article}
\usepackage{ctex}
\usepackage{geometry}
\usepackage{graphicx}
\usepackage{float}
\usepackage{listings}
\usepackage{xcolor}
\lstset{
    numbers=left, 
    numberstyle= \tiny, 
    keywordstyle= \color{ blue!70},
    commentstyle= \color{red!50!green!50!blue!50}, 
    frame=shadowbox,
    rulesepcolor= \color{ red!20!green!20!blue!20} ,
    escapeinside=``,
    framexleftmargin=2em
}

\geometry{left=3.0cm,right=3.0cm,top=3.5cm,bottom=3.5cm}

\title{点亮数字人生}
\author{2017011341, 陈旭}
\date{2019 年 4 月}

\begin{document}

\maketitle

\section{实验目的}

    \begin{itemize}
        \item 通过数码管点亮程序,熟悉 VHDL 语言,了解掌握硬件程序的编写规范;
        \item 进一步理解可编程芯片的工作原理。
    \end{itemize}

\section{实验内容}

    \begin{itemize}
        \item 同时点亮一个经过译码的数码管和一个未经过译码的数码管;
        \item 设计一个数码管显示实验,要求有规律地显示数列(奇数列、偶数列、自然数列等),尽可能多地点亮数码管。要求试验中至少使用一个不带译码的数码管。
    \end{itemize}

\section{代码与分析}

    \subsection{实验内容 1}

        \subsubsection{代码}

        \begin{lstlisting}
LIBRARY IEEE;
USE IEEE.STD_LOGIC_1164.ALL;
USE IEEE.STD_LOGIC_ARITH.ALL;
USE IEEE.STD_LOGIC_UNSIGNED.ALL;
entity test is
    port(
        key: in std_logic_vector(3 downto 0);
        display: out std_logic_vector(6 downto 0);
        display_4: out std_logic_vector(3 downto 0)
    );
end test;
architecture fire of test is
begin
    display_4<=key;
    process(key)
    begin
        case key is
            when "0000"=>display<="1111110";
            when "0001"=>display<="0110000";
            when "0010"=>display<="1101101";
            when "0011"=>display<="1111001";
            when "0100"=>display<="0110011";
            when "0101"=>display<="1011011";
            when "0110"=>display<="0011111";
            when "0111"=>display<="1110000";
            when "1000"=>display<="1111111";
            when "1001"=>display<="1110011";
            when others=>display<="0000000";
        end case;
    end process;
end fire;
            \end{lstlisting}

        \subsubsection{分析}
            \par 设置 display 用于控制不带译码器的数码管,display\_4 用于控制带译码器的数码管,key 表示开关的输入端口,通过 4 个开关来控制 4 个 key 的值,数码管显示相应的值。

    \subsection{实验内容 2}

        \subsubsection{代码}

        \begin{lstlisting}
LIBRARY IEEE;
USE IEEE.STD_LOGIC_1164.ALL;
USE IEEE.STD_LOGIC_ARITH.ALL;
USE IEEE.STD_LOGIC_UNSIGNED.ALL;
entity test is
    port(
        display: out std_logic_vector(6 downto 0);
        display_4_oe: out std_logic_vector(2 downto 0);
        oe: out std_logic;
        clk: in std_logic;
        rst: in std_logic;
        btn: in std_logic
    );
end test;
architecture fire of test is
    signal display_4_buf: std_logic_vector(3 downto 0):="0000";
    signal display_4_buf_oe: std_logic_vector(2 downto 0):="000";
    signal cnt:integer:=0;
    signal j:integer range 0 to 10:=1;
    signal rst1:boolean:=false;
    signal rst2:boolean:=false;
    signal btn1:boolean:=false;
    signal btn2:boolean:=false;
    signal go:boolean:=true;
begin
    process(clk)
    begin
        if (clk'event and clk='1') then
            if (rst1 = NOT rst2) then
                display_4_buf<="0000";
                display_4_buf_oe<="000";
                j<=1;
                rst2<=rst1;
                oe<='0';
                cnt<=-1000000;
                go<=true;
            else
                if (btn1 = NOT btn2) then
                    go <= NOT go;
                    btn2 <= btn1;
                end if;
                if (go) then
                    cnt<=cnt+1;
                    if (cnt=500000) then
                        oe<='1';
                    end if;
                    if (cnt=1000000) then
                        cnt<=0;
                        j<=j+1;
                        if (j=5) then
                            j<=1;
                            if (display_4_buf="1001") then
                                display_4_buf<="0000";
                            else
                                display_4_buf<=display_4_buf+1;
                            end if;
                        end if;
                        if (display_4_buf_oe="100") then
                            display_4_buf_oe<="000";
                        else
                            display_4_buf_oe<=display_4_buf_oe+1;
                        end if;
                        oe<='0';
                    end if;
                end if;
            end if;
            display_4_oe<=display_4_buf_oe;
        end if;
    end process;
    process(rst)
    begin
        if (rst'event and rst='1') then
            rst1<=NOT rst2;
        end if;
    end process;
    process(btn)
    begin
        if (btn'event and btn='1') then
            btn1<=NOT btn2;
        end if;
    end process;
    process(display_4_buf)
    begin
        case display_4_buf is
            when "0000"=>display<="1111110";
            when "0001"=>display<="0110000";
            when "0010"=>display<="1101101";
            when "0011"=>display<="1111001";
            when "0100"=>display<="0110011";
            when "0101"=>display<="1011011";
            when "0110"=>display<="1011111";
            when "0111"=>display<="1110000";
            when "1000"=>display<="1111111";
            when "1001"=>display<="1111011";
            when others=>display<="0000000";
        end case;
    end process;
end fire;
            \end{lstlisting}

        \subsubsection{分析}
            \par 这里控制了一个带译码器的数码管显示自然数列,另外两个带译码器的数码管将 2、4、8 端口与该译码器接在一起,1 端口分别接入高电平和低电平,这样就能轻易实现自然数列、奇数列、偶数列的显示了。不带译码器的数码管额外处理。
            \par 该代码还实现了通过按键来控制重置与暂停继续的功能(数列随着时间自动增长)。


\end{document}