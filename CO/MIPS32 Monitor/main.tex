%!TEX program=xelatex
\documentclass[a4paper]{article}
\usepackage{ctex}
\usepackage{geometry}
\usepackage{graphicx}
\usepackage{float}
\usepackage{listings}
\usepackage{xcolor}
\lstset{
    numbers=left, 
    numberstyle= \tiny, 
    keywordstyle= \color{ blue!70},
    commentstyle= \color{red!50!green!50!blue!50}, 
    frame=shadowbox,
    rulesepcolor= \color{ red!20!green!20!blue!20} ,
    escapeinside=``,
    framexleftmargin=2em
}
\geometry{left=3.0cm,right=3.0cm,top=3.5cm,bottom=3.5cm}
\title{MIPS32监控程序模拟实验}
\author{2017011341, 陈旭}
\date{2019 年 9 月}
\begin{document}
\maketitle
\newpage
\section{熟悉环境}

    \subsection{启动模拟器}

    \begin{itemize}
        \item 运行 ... 启动 MIPS32 模拟器;
        \item 运行 ... 连接 MIPS32 模拟器,然后在弹出的命令行中执行相关指令。
    \end{itemize}

    \subsection{A 指令}

        用于向内存中写入汇编代码,使用方法:在命令行中输入 A,并回车,接着输入需要写入的内存地址,然后即可开始编写汇编代码,以空行结束。

        \begin{figure}[H]
            \centering
            %\includegraphics[width=0.7\linewidth]{figures/}
            \caption{A 指令示例}
        \end{figure}

    \subsection{G 指令}

        编译并运行指定地址上的汇编代码,使用方法:在命令行输入 G,并回车,接着输入汇编代码所在的内存地址。

        \begin{figure}[H]
            \centering
            %\includegraphics[width=0.7\linewidth]{figures/}
            \caption{G 指令示例}
        \end{figure}

    \subsection{D 指令}

        查看内存上的数据,使用方法:在命令行输入 D,并回车,接着输入需要查看的内存地址,以及查看的长度。

        \begin{figure}[H]
            \centering
            %\includegraphics[width=0.7\linewidth]{figures/}
            \caption{D 指令示例}
        \end{figure}

    \newpage


\section{实验内容}

    \subsection{斐波那契数列}

        \subsubsection{实验说明}

            求前 10 个 Fibonacci 数,将结果保存到起始地址为 0x80400000 的 10 个字中,并用 D 命令检查结果正确性。
        
        \subsubsection{实验思路}

            实验主体是实现一个循环,在循环中通过三个变量之间的轮转来求斐波那契数列,同时将数写入内存中,每次循环内存地址和计数变量都增加(内存地址每次增加 4),计数变量到 10 时退出循环。
        
        \subsubsection{实验代码}

            \begin{lstlisting}
li $t0, 0x1
li $t1, 0x1
li $t3, 0x0
li $t4, 0xA
li $a0 , 0x80400000
sw $t0, 0($a0)
addio $a0, $a0, 4
addu $t2, $t0, $t1
move $t0, $t1
move $t1, $t2
addiu $t3, $t3, 0x1
bne $t3, $t4, -28
nop
jr $31
nop
            \end{lstlisting}

        \subsubsection{实验结果}

            \begin{figure}[H]
                \centering
                %\includegraphics[width=0.7\linewidth]{figures/}
                \caption{任务 1 实验结果}
            \end{figure}

        \newpage

    \subsection{可见字符}

        \subsubsection{实验说明}
        
            输出所有可见字符(0x21$\sim$0x7F)到终端。
        
        \subsubsection{实验思路}

            实验主体是实现一个循环,从 0x21 开始将对应的字符输出,每次循环对输出的 ASCII 码进行累加,大于 0x7F 则退出。
        
        \subsubsection{实验代码}

            \begin{lstlisting}
li $t0, 0x1
li $t1, 0x1
li $t3, 0x0
li $t4, 0xA
li $a0 , 0x80400000
sw $t0, 0($a0)
addio $a0, $a0, 4
addu $t2, $t0, $t1
move $t0, $t1
move $t1, $t2
addiu $t3, $t3, 0x1
bne $t3, $t4, -28
nop
jr $31
nop
            \end{lstlisting}

        \subsubsection{实验结果}

            \begin{figure}[H]
                \centering
                %\includegraphics[width=0.7\linewidth]{figures/}
                \caption{任务 2 实验结果}
            \end{figure}

        \newpage

    \subsection{斐波那契数}

        \subsubsection{实验说明}
        
            求第 60 个 Fibonacci 数,将结果保存到起始地址为 0x80400000 的 8 个字节中,并用 D 命令检查结果正确性。
        
        \subsubsection{实验思路}

            整体思路同任务 1,难点在于数值较大,计算时需要将数字拆分为高 32 位和低 32 位分别计算,特别处理的地方在于低位计算溢出时需要向高位进位,即高位需要加上 1。
        
        \subsubsection{实验代码}

            \begin{lstlisting}
li $t0, 0x1
li $t1, 0x1
li $t3, 0x0
li $t4, 0xA
li $a0 , 0x80400000
sw $t0, 0($a0)
addio $a0, $a0, 4
addu $t2, $t0, $t1
move $t0, $t1
move $t1, $t2
addiu $t3, $t3, 0x1
bne $t3, $t4, -28
nop
jr $31
nop
            \end{lstlisting}
            
        \subsubsection{实验结果}

            \begin{figure}[H]
                \centering
                %\includegraphics[width=0.7\linewidth]{figures/}
                \caption{任务 3 实验结果}
            \end{figure}

        \newpage

\section{实验总结}
    
    本次实验内容对于平常的编程来说都是属于比较简单的题目,而实验目的主要也是为了熟悉 MIPS32 指令,在实验的同时也自己琢磨了一些 Debug 的方法,相信这次实验对之后工作会有很大的帮助。

\end{document}